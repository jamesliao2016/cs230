\documentclass{article} % For LaTeX2e
\usepackage{enumitem,amssymb,latexsym,amsmath,graphicx,amsthm,xkeyval,subfig,siunitx,booktabs}
\usepackage{nips13submit_e,times}
\usepackage{hyperref}
\usepackage{url}
\usepackage{pythonhighlight}
%\documentstyle[nips13submit_09,times,art10]{article} % For LaTeX 2.09

\title{CS230 Stock Price Prediction (Milestone)}

\author{
Ryan Almodovar \\
Stanford University\\
\texttt{ralmodov@stanford.edu} \\
}

\newcommand{\fix}{\marginpar{FIX}}
\newcommand{\new}{\marginpar{NEW}}
\newcommand{\todo}[1]{\textbf{TODO: {#1}}}
\newcommand\fnurl[2]{%
\href{#2}{#1}\footnote{\url{#2}}%
}

\nipsfinalcopy % Uncomment for camera-ready version

\begin{document}

\maketitle

% TODO
% \begin{abstract}
% \end{abstract}

\section{Introduction}
Generalized case of predicting the Dow Jones Industrial Average (DJIA), then focus on specific companies found in headlines
for more accurate results.
Due to time constraints, the baseline model for this milestone just contains the general DJIA predictions though the
specific companies are planned to also be included in the finalized report.

\section{Dataset Details}
For DJIA, a dataset the with a top 25 of news headlines extracted from news articles.
The current data was provided through Kaggle (https://www.kaggle.com/aaron7sun/stocknews) and included data points over a period of roughly 8 years.

\section{Approach}
Split the dataset train, dev, and test sets with distributions 80\%, 10\%, and 10\% respectively.
Using NLTK's Sentiment Intensity Analyzer as a simple baseline model, it determined the average sentiment of the top 25 news headlines and scored a 55\% prediction rate. 
Some other architectures 
The main issue with this approach is that the average sentiment may result in a loss of information which could be a reason for poor accuracy.
In order to improve the prediction rate, I plan to gather more data specific to particular companies/industries and predict the stock prices of these rather than only the generalized DJIA, as well as improving the architecture rather than the Sentiment Analyzer.
implement and experiment using LSTM/GRU cells, as well as the Neural Tensor Network (NTN) described in Ding et al 2015 (https://www.ijcai.org/Proceedings/15/Papers/329.pdf) for event-driven embedding.
For the NTN to be implemented, each headline needs to be transformed into an event tuple described in the NTN architecture.
Some candidate options to be able to transform the headline tokens into the event tuple are to use SpaCy's library
I further plan to expand the dataset and combine it with 

\section{Baseline Code}
\begin{python}
def f(x):
    return x
\end{python}


\end{document}
