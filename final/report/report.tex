\documentclass{article} % For LaTeX2e
\usepackage{enumitem,amssymb,latexsym,amsmath,graphicx,amsthm,xkeyval,subfig,siunitx,booktabs}
\usepackage{nips13submit_e,times}
\usepackage{hyperref}
\usepackage{url}
%\documentstyle[nips13submit_09,times,art10]{article} % For LaTeX 2.09

\title{CS230 Stock Price Prediction}

\author{
Ryan Almodovar \\
Stanford University\\
\texttt{ralmodov@stanford.edu} \\
}

\newcommand{\fix}{\marginpar{FIX}}
\newcommand{\new}{\marginpar{NEW}}
\newcommand{\todo}[1]{\textbf{TODO: {#1}}}
\newcommand\fnurl[2]{%
\href{#2}{#1}\footnote{\url{#2}}%
}

\nipsfinalcopy % Uncomment for camera-ready version

\begin{document}

\maketitle

% TODO
% \begin{abstract}
% \end{abstract}

\section{Introduction}
Predicting the Dow Jones Industrial Average (DJIA) with a top 25 of news headlines extracted from news articles.
this section introduces your project, why it’s important or interesting.

\section{Baseline Code}

\section{Dataset Details}
Data was provided through Kaggle (https://www.kaggle.com/aaron7sun/stocknews) and included data points over a period of roughly 8 years.

\section{Approach}
Split the dataset train, dev, and test sets with distributions 80\%, 10\%, and 10\% respectively.
Currently the baseline model uses NLTK's SentimentIntensityAnalyzer to determine the average sentiment of the top 25 news headlines and achieves
Plan to use SpaCy to create event tuples for input to Neural Tensor Network


\end{document}
